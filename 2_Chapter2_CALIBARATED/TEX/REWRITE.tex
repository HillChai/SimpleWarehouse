% \documentclass[a4paper, 11pt]{article}
\documentclass{article}

\usepackage[dvipsnames]{xcolor}
\usepackage{ctex}
\usepackage{graphicx}
\usepackage[unicode]{hyperref}
\usepackage{cite}
\usepackage{indentfirst}
\usepackage{listings}
\usepackage{geometry}
\usepackage{amsmath}
\usepackage{tabularx}

\geometry{a4paper, left=2cm, right=2cm, top=1cm, bottom=1.5cm}

% \lstset{
%   language=C++,
%   basicstyle=\fontsize{8}{5}\ttfamily, % 设置代码字体大小为 12pt
%   breaklines=true, % 自动换行
%   numbers=left,
%   showstringspaces = false
% }

%%%%%% 设置字号 %%%%%% 
\newcommand{\chuhao}{\fontsize{nn42pt}{\baselineskip}\selectfont}
\newcommand{\xiaochuhao}{\fontsize{36pt}{\baselineskip}\selectfont}
\newcommand{\yihao}{\fontsize{28pt}{\baselineskip}\selectfont}
\newcommand{\erhao}{\fontsize{21pt}{\baselineskip}\selectfont}
\newcommand{\xiaoerhao}{\fontsize{18pt}{\baselineskip}\selectfont}
\newcommand{\sanhao}{\fontsize{15.75pt}{\baselineskip}\selectfont}
\newcommand{\sihao}{\fontsize{14pt}{\baselineskip}\selectfont}
\newcommand{\xiaosihao}{\fontsize{12pt}{\baselineskip}\selectfont}
\newcommand{\wuhao}{\fontsize{10.5pt}{\baselineskip}\selectfont}
\newcommand{\xiaowuhao}{\fontsize{9pt}{\baselineskip}\selectfont}
\newcommand{\liuhao}{\fontsize{7.875pt}{\baselineskip}\selectfont}
\newcommand{\qihao}{\fontsize{5.25pt}{\baselineskip}\selectfont}

% %%%% 设置 section 属性 %%%%
% \makeatletter
% \renewcommand\section{\@startsection{section}{1}{\z@}%
% {-1.5ex \@plus -.5ex \@minus -.2ex}%
% {.5ex \@plus .1ex}%
% {\normalfont\sihao\CJKfamily{hei}}}
% \makeatother

 %%%% 设置 subsection 属性 %%%%
% \makeatletter
% \renewcommand\subsection{\@startsection{subsection}{1}{\z@}%
% % {-1.25ex \@plus -.5ex \@minus -.2ex}%
% % {-1ex \@plus -.5ex \@minus -.2ex}%
% {-1ex \@plus -.3ex \@minus -.1ex}%
% {.4ex \@plus .1ex}%
% {\normalfont\xiaosihao\CJKfamily{hei}}}
% \makeatother

% %%%% 设置 subsubsection 属性 %%%%
% \makeatletter
% \renewcommand\subsubsection{\@startsection{subsubsection}{1}{\z@}%
% {-1ex \@plus -.5ex \@minus -.2ex}%
% {.3ex \@plus .1ex}%
% {\normalfont\xiaosihao\CJKfamily{hei}}}
% \makeatother

%%%% 段落首行缩进两个字 %%%%
\makeatletter
\let\@afterindentfalse\@afterindenttrue
\@afterindenttrue
\makeatother
% \setlength{\parindent}{2em}  %中文缩进两个汉字位

%%%% 下面的命令重定义页面边距,使其符合中文刊物习惯 %%%%
% \addtolength{\topmargin}{-54pt}
% \setlength{\oddsidemargin}{0.63cm}  % 3.17cm - 1 inch
% \setlength{\evensidemargin}{\oddsidemargin}
% \setlength{\textwidth}{14.66cm}
% \setlength{\textheight}{24.00cm}    % 24.62

%%%% 下面的命令设置行间距与段落间距 %%%%
\linespread{1.0}
% \setlength{\parskip}{1ex}
\setlength{\parskip}{0.5\baselineskip}

% 在导言区进行样式设置
\lstset{
    language=C++, % 设置语言
 	basicstyle=\ttfamily, % 设置字体族
 	breaklines=true, % 自动换行
 	keywordstyle=\bfseries\color{NavyBlue}, % 设置关键字为粗体,颜色为 NavyBlue
 	morekeywords={PressureSensor, Button, Oled}, % 设置更多的关键字,用逗号分隔
 	emph={self}, % 指定强调词,如果有多个,用逗号隔开
    emphstyle=\bfseries\color{Rhodamine}, % 强调词样式设置
    commentstyle=\itshape\color{black!50!white}, % 设置注释样式,斜体,浅灰色
    stringstyle=\bfseries\color{PineGreen!90!black}, % 设置字符串样式
    columns=flexible,
    numbers=left, % 显示行号在左边
    numbersep=2em, % 设置行号的具体位置
    numberstyle=\footnotesize, % 缩小行号
    % frame=single, % 边框
	tabsize = 4,  %行缩进
    framesep=1em % 设置代码与边框的距离
}

%%%% 正文开始 %%%%
\begin{document}	
		%%%% 定理类环境的定义 %%%%
\newtheorem{example}{例}             % 整体编号
\newtheorem{algorithm}{算法}
\newtheorem{theorem}{定理}[section]  % 按 section 编号
\newtheorem{definition}{定义}
\newtheorem{axiom}{公理}
\newtheorem{property}{性质}
\newtheorem{proposition}{命题}
\newtheorem{lemma}{引理}
\newtheorem{corollary}{推论}
\newtheorem{remark}{注解}
\newtheorem{condition}{条件}
\newtheorem{conclusion}{结论}
\newtheorem{assumption}{假设}

		%%%% 重定义 %%%%
\renewcommand{\contentsname}{目录}  % 将Contents改为目录
\renewcommand{\abstractname}{摘要}  % 将Abstract改为摘要
\renewcommand{\refname}{参考文献}   % 将References改为参考文献
\renewcommand{\indexname}{索引}
\renewcommand{\figurename}{图}
\renewcommand{\tablename}{表}
\renewcommand{\appendixname}{附录}
\renewcommand{\algorithm}{算法}	

		%%% 定义标题格式,包括title,author,affiliation,email等 %%%%
\title{智能仓储系统的开发研究}
% \author{XXX\footnote{电子邮件: XXXXXXXXXXXX@zjut.edu.cn 学号: XXXXXXXXXXXX}\\[2ex]
% \xiaosihao 浙江工业大学\\[2ex]}
\author{\xiaosihao 先进计算与机器人研究所}
%\date{}
		
		%%%% 以下部分是正文 %%%%  
\maketitle
		
\tableofcontents
\newpage

\section{第二章: 等待校准的秤}
从上一章1.3节的图像可以看出, 称重结果有些不准, 离真实数据偏差较大, 我们需要进一步校准。从图像看出, 用直线拟合效果也能接接受。
在这一章我们将给出一个可以进行一次校准的秤, 还是在1.3节Calibrate类上进行修改, 加入kb在arduino硬盘上读写的操作, 以及将校准后的结果打印在串口。

秤校准的步骤和1.3节相比基本类似:
\begin{itemize}
  \item 1. 获得不准的数据数组;
  \item 2. 计算拟合系数;
  \item 3. 输出校准后的结果
\end{itemize}

\begin{lstlisting}
class Calibrate {
union coeffience {
  unsigned long value_out;
  byte value_in[4];
};

private:
  int pin_SCK, pin_DT, range; 
  float GapValue;
  unsigned long k, b;
  unsigned long x[13];
  int n;

public:
  Calibrate();

  void Get_x_array(unsigned long ADS);
  void Get_kb();
  unsigned long Output_CalibratedWeight(unsigned long weight);
  void kb_Initialize();

  void setpin_SCKDT(int p1, int p2);
  void set_range(int r);
  unsigned long Output_Weight(unsigned long ADS);
  unsigned long HX711_Read();
};
\end{lstlisting}

相比1.3节添加了两个函数, 修改了一个函数。
\begin{itemize}
  \item Get\_kb中加入了写入arduino硬盘的操作;
  \item Output\_CalibratedWeight输出校准后的读数;
  \item kb\_Initialize是从arduino硬盘上读取存的k,b值。
\end{itemize}

\begin{lstlisting}
void Calibrate::Get_kb(){
  unsigned long y[n] = {5, 15, 35, 55, 105, 574}, sum_x = 0, sum_y = 0;
  for (unsigned long& yi:y) 
      sum_y+=yi;
  for (unsigned long& x_i:x) 
    sum_x+=x_i;
  unsigned long mean_x = sum_x/n, mean_y = sum_y/n;
  unsigned long k1=0, k2=0;
  for (int i=0; i<n; ++i) {
    k1+=x[i]*y[i];
    k2+=x[i]*x[i];
  }
  k = (k1-n*mean_x*mean_y)/(k2-n*mean_x*mean_x);
  b = mean_y - mean_x * k;

  coeffience k_out;
  coeffience b_out;

  k_out.value_out = k;
  b_out.value_out = b;

  for(int i=4; i<8; i++)    //判断是否与上次储存的k相同
    if (!(EEPROM.read(i)==k_out.value_in[i-4])) {
      Serial.print("newk:");
      Serial.println(k);
      for (int j=i; j<8; j++) 
        EEPROM.write(j, k_out.value_in[j-4]);
      break;
  }
        
  for(int i=8; i<12; i++)    //判断是否与上次储存的b相同
    if (!(EEPROM.read(i)==b_out.value_in[i-8])) {
      Serial.print("newb:");
      Serial.println(b);
      for (int j=8; j<12; j++) 
        EEPROM.write(j, b_out.value_in[j-8]);
      break;
    }
};
\end{lstlisting}

在得到k,b后,用共用体k\_out, b\_out存储k, b值, 通过for循环与arduino上对应地址存的k(4~7), b(8~11)值比较, 
如果不一样则从当前位置写入。

\begin{lstlisting}
coeffience k_out;
coeffience b_out;

k_out.value_out = k;
b_out.value_out = b;

for(int i=4; i<8; i++)    //判断是否与上次储存的k相同
  if (!(EEPROM.read(i)==k_out.value_in[i-4])) {
    Serial.print("newk:");
    Serial.println(k);
    for (int j=i; j<8; j++) 
      EEPROM.write(j, k_out.value_in[j-4]);
    break;
}
        
for(int i=8; i<12; i++)    //判断是否与上次储存的b相同
  if (!(EEPROM.read(i)==b_out.value_in[i-8])) {
    Serial.print("newb:");
    Serial.println(b);
    for (int j=8; j<12; j++) 
      EEPROM.write(j, b_out.value_in[j-8]);
    break;
  }
\end{lstlisting}

校准结果是在第一章中Output\_Weight基础上, 加入了系数k,b的作用, 返回校准后的质量。
\begin{lstlisting}
unsigned long Calibrate::Output_CalibratedWeight(unsigned long ADS){
  return k*Output_Weight(ADS)+b;
};
\end{lstlisting}

如果之前校准过, 可以通过kb\_Initialize()读取arduino上存的k,b值:
\begin{lstlisting}
void PressureSensor::kb_Initialize() {
  coeffience k_arduino;
  coeffience b_arduino;
  
  for(int i=4; i<8; i++) {
    k_arduino.value_in[i-4] = EEPROM.read(i);
    b_arduino.value_in[i-4] = EEPROM.read(i+4);
  }

  k = k_arduino.value_out;
  b = b_arduino.value_out;
  Serial.print("k: ");
  Serial.println(k);
  Serial.print("b: ");
  Serial.println(b);
};
\end{lstlisting}

arduino上k对应的地址是4到7, b对应的地址是8到11。首先创建两个共用体k\_arduino, b\_arduino。
\begin{lstlisting}
  coeffience k_arduino;
  coeffience b_arduino; 
\end{lstlisting}

然后用一个for循环读对应地址上信息存放到对应共用体中。
\begin{lstlisting}
  for(int i=4; i<8; i++) {
    k_arduino.value_in[i-4] = EEPROM.read(i);
    b_arduino.value_in[i-4] = EEPROM.read(i+4);
  }  
\end{lstlisting}

最后是更新变量k,b, 并在串口打印。
\begin{lstlisting}
  k = k_arduino.value_out;
  b = b_arduino.value_out;
  Serial.print("k: ");
  Serial.println(k);
  Serial.print("b: ");
  Serial.println(b);
\end{lstlisting}

\subsection{Calibrate.ino}
\begin{lstlisting}
#include "Surface.h"
#include "Calibrate.h"

Surface YL_Surface;
Calibrate YL_Calibrated;

bool flag=1;

void setup() {
  Serial.begin(9600);
  YL_Calibrated.setpin_SCKDT(4, 5);
  YL_Calibrated.set_range(20);
  YL_Calibrated.kb_Initialize();
}

void loop() {
  if (flag) {
    YL_Calibrated.Get_x_array(YL_Surface.Get_Surface());
    YL_Calibrated.Get_kb();
    flag=0;
  }

  unsigned long CalibratedWeight = YL_Calibrated.Output_CalibratedWeight(YL_Surface.Get_Surface());
  Serial.println(CalibratedWeight);
  delay(1000);
}
\end{lstlisting}

主程序分为三部分:

\noindent 第一部分: 导入头文件Calibrate.h, Surface.h, 实例化Surface类. Oled类, flag=1方便主循环调用一次校准;
\begin{lstlisting}
  #include "Surface.h"
  #include "Calibrate.h"
  
  Surface YL_Surface;
  Calibrate YL_Calibrated;
  
  bool flag=1;
\end{lstlisting}  

\noindent 第二部分: 在setup函数里初始化波特率9600, 与Arduino串口通信的波特率保持一致; 
\begin{lstlisting}
  Serial.begin(9600);
\end{lstlisting}

设置与SCK和DT相连的arduino引脚4和5, 设置当前压力传感器的量程20. kb\_Initialize()读取之前存的k,b值:
\begin{lstlisting}
  YL_Calibrated.setpin_SCKDT(4, 5);
  YL_Calibrated.set_range(20);
  YL_Calibrated.kb_Initialize();
\end{lstlisting}

\noindent 第三部分: flag为1时进行一次校准,并在校准后flag赋值为0, 避免重复校准。 
先通过Get\_x\_array获得不准的读数数组, 再通过Get\_kb获得拟合系数, 最后由Output\_CalibratedWeight计算校准后的结果。
\begin{lstlisting}
if (flag) {
  YL_Calibrated.Get_x_array(YL_Surface.Get_Surface());
  YL_Calibrated.Get_kb();
  flag=0;
}

unsigned long CalibratedWeight = YL_Calibrated.Output_CalibratedWeight(YL_Surface.Get_Surface());
Serial.println(CalibratedWeight);
delay(1000);
}
\end{lstlisting}

\end{document} 

